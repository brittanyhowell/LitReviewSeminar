\documentclass{beamer}
\usepackage[export]{adjustbox} % For figure frames
\usepackage{graphicx} % For figures
\usepackage{xmpmulti} % for gifs
\usepackage{animate}
\usepackage{natbib}

% Theme
\usetheme{Dresden}
%\usetheme{Malmoe} 
%\usetheme{Montpellier}

% Colour
\usecolortheme{beaver} % red and light grey
%\usecolortheme{dolphin} %lighter blue than normal
%\usecolortheme{dove} % Black and White
%\usecolortheme{seahorse} % Pale purple

\setbeamercolor{institute in head/foot}{parent=palette primary}

\AtBeginSection[]{
	\begin{frame}
		\vfill
		\centering
		\begin{beamercolorbox}[sep=8pt,center]{title}
			\usebeamerfont{title}\insertsectionhead\par%
		\end{beamercolorbox}
		\vfill
	\end{frame}}

\title{Exploring the possibility of Alternative Splicing as a path to the regulation of LINE-1 elements in human and mouse}
\author{Brittany Howell}
\institute{Supervisors: Prof. Dave Adelson and Dr. Dan Kortschak}
\date{March 17, 2016}

\begin{document}
	
	\frame{\titlepage}

		\begin{frame} % Overview
			
			\frametitle{Overview}
			
			\begin{itemize}
				\item Background: Transposable Elements (TEs) including L1s
			\end{itemize}
			\begin{columns}[c] % Columns are left aligned
				
				\begin{column}[T]{.6\textwidth}
					
					\begin{figure}
						\includegraphics[width=\textwidth]{../../Figures/genome-puzzle.png}
					\end{figure}
					
				\end{column}
			
			\begin{column}[T]{.55\textwidth}
				\begin{itemize}
					\item Regulation of TEs
					\item Alternative splicing
					\item Alternative splicing in L1s
				\end{itemize}
			\end{column}
			
				
			
			\end{columns}
				\end{frame}
			
			\section[Background]{Background}
			
		\begin{frame} % Exons, Introns and Other
			
			\frametitle{The human genome}
			\framesubtitle{Repetitive elements are abundant in the human genome}
			
			\begin{columns}
				\begin{column}{0.6\linewidth}
					
					\begin{figure}
						\includegraphics[scale=0.3]{../../Figures/Seminar-genome-All.png}			
					\end{figure}
				\end{column}
			
				\begin{column}{0.4\linewidth}
					\begin{block}{Other genome content:}
						Tandem repeats\\
						Intergenic regions\\
						Duplications\\
						Transposable elements
					\end{block}
				\end{column}
			\end{columns}
	
			
			Xu et al. 2010, Singer et al. 2010

		\end{frame}
		
		\begin{frame} % Proportion of T Es
			
			\frametitle{The human genome}
			\framesubtitle{Repetitive elements are abundant in the human genome}
			
					\begin{figure}
						\includegraphics[scale=0.3]{../../Figures/Seminar-genome-TE.jpg}
					\end{figure}
						Xu et al. 2010, Singer et al. 2010
		\end{frame}
		
		\begin{frame} % Proportion of L1s
			
			\frametitle{The human genome}
			\framesubtitle{Repetitive elements are abundant in the human genome}
			

			\begin{figure}
				\includegraphics[scale=0.3]{../../Figures/Seminar-genome-L1s.png}
			\end{figure}
			
			Xu et al. 2010, Singer et al. 2010

		\end{frame}
		
		\begin{frame} % L1s description
			
			\frametitle{LINE-1 (L1) structure}
			\framesubtitle{L1s are TEs}
			
			\begin{figure}
				\includegraphics[scale=0.4]{../../Figures/L1-structure.jpg}
			\end{figure}
			
			\begin{block}{L1 structure}
				\begin{itemize}
					\item Full length L1s are 6-7kb 
					\item L1s are often 5' truncated, inverted or degraded
					\item Some variants are 4kb - HAL1s
				\end{itemize}
				
			\end{block}
		\end{frame}
		
		\begin{frame} % Retrotransposition 1
			
			\frametitle{TE replication cycle}
			\framesubtitle{LINE-1s are retrotransposons}

			\begin{figure}
				\includegraphics[scale=0.25]{../../Figures/L1-RT-1.png}
			\end{figure}
			\begin{itemize}
				\item L1s replicate through an RNA intermediate
				\item They integrate anywhere in the genome - interspersed repeats
			\end{itemize}
			
		

		\end{frame}
	
		\begin{frame} % Retrotransposition Transcription
			
			\frametitle{TE replication cycle}
			\framesubtitle{LINE-1s are retrotransposons}
			
			\begin{figure}
				\includegraphics[scale=0.25]{../../Figures/L1-RT-2.png}
			\end{figure}
			\begin{itemize}
				\item L1s replicate through an RNA intermediate
				\item They integrate anywhere in the genome - interspersed repeats
			\end{itemize}
			
		\end{frame}
		
		\begin{frame} % Retrotransposition Translation
			
			\frametitle{TE replication cycle}
			\framesubtitle{LINE-1s are retrotransposons}
			
			\begin{figure}
				\includegraphics[scale=0.25]{../../Figures/L1-RT-3.png}
			\end{figure}
			\begin{itemize}
				\item L1s replicate through an RNA intermediate
				\item They integrate anywhere in the genome - interspersed repeats
			\end{itemize}
			
		\end{frame}
		
		\begin{frame} % Retrotransposition Association
			
			\frametitle{TE replication cycle}
			\framesubtitle{LINE-1s are retrotransposons}
			
			\begin{figure}
				\includegraphics[scale=0.25]{../../Figures/L1-RT-4.png}
			\end{figure}
			\begin{itemize}
				\item L1s replicate through an RNA intermediate
				\item They integrate anywhere in the genome - interspersed repeats
			\end{itemize}
			
		\end{frame}
		
		\begin{frame} % Retrotransposition Integration
			
			\frametitle{TE replication cycle}
			\framesubtitle{LINE-1s are retrotransposons}
			
			\begin{figure}
				\includegraphics[scale=0.25]{../../Figures/L1-RT-5.png}
			\end{figure}
			\begin{itemize}
				\item L1s replicate through an RNA intermediate
				\item They integrate anywhere in the genome - interspersed repeats
			\end{itemize}
			
		\end{frame}
		
		\begin{frame} % Retrotransposition Summary
			
			\frametitle{TE replication cycle}
			\framesubtitle{LINE-1s are retrotransposons}
			
			\begin{figure}
				\includegraphics[scale=0.25]{../../Figures/L1-RT-All.png}
			\end{figure}
			\begin{itemize}
				\item L1s replicate through an RNA intermediate
				\item They integrate anywhere in the genome - interspersed repeats
			\end{itemize}
			
		\end{frame}
			
			\section[Regulation of TEs]{Regulation of TEs}
	
		\begin{frame} % Regulation intro
			\frametitle{Regulation of TEs}
			\framesubtitle{Why is regulation required?}
			
%				\begin{columns}[c]
					
%					\begin{column}{.3\textwidth}
%						
%						\begin{itemize}
%							\item TEs are highly mutagenic
%							\item ADD 
%						\end{itemize}	
%						
%					\end{column}
					
%					\begin{column}{.7\textwidth}
						
						\begin{figure}
							\includegraphics[width=0.93\linewidth]{../../Figures/L1-insertion.png}
						\end{figure}	
						
%					\end{column}
%					
%				\end{columns}
				
				
		\end{frame}		
	
		\begin{frame} % Methylation
			
			\frametitle{DNA methylation}
			\framesubtitle{Methylation of DNA is widespread throughout the genome}
			\begin{columns}
	
				\begin{column}{0.5\linewidth}
		
			\begin{itemize}
%				\item Methylation occurs throughout the genome \textit{e.g.} X chromosome inactivation, and TE silencing
				\item X chromosome inactivation, TE silencing
%				\item Thought to be the mechanism of TE regulation - absence of methylation correlated with TE accumulation
				\item Absence of methylation occurs with TE accumulation 
%				\item DNA methylation levels decrease in primordial germ cells, providing TEs opportunity for replication
				\item Levels fluctuate in development
				\item L1s in the female mature gamete aren't fully methylated
			\end{itemize}
				\end{column}
				\begin{column}{0.5\linewidth}
					\begin{figure}
						\centering
						\includegraphics[width=\linewidth]{../../Figures/Methylation.png}
					\end{figure}
				\end{column}
			\end{columns}
		\end{frame}		
		
		\begin{frame} % Other regulation
			
			\frametitle{Other regulation}
			\framesubtitle{Many other mechanisms have been shown to suppress TEs}
			\begin{columns}
				\begin{column}{0.4\linewidth}
					\begin{itemize}
						\item Histone modifications
						\begin{itemize}
							\item Methylation - SETB1
							\item Ubiquitination
							\item Acetylation
						\end{itemize}
						\item RNA interference
						\begin{itemize}
							\item miRNAs
							\item siRNAs
							\item piRNAs
						\end{itemize}
						\item RNA editases
						\begin{itemize}
							\item APOBEC
						\end{itemize}
					\end{itemize}
				\end{column}
				\begin{column}{0.6\linewidth}
					\begin{figure}
						\includegraphics[width=0.6\linewidth]{../../Figures/Epigenetics-title.png}
					\end{figure}
					\end{column}
			\end{columns}
		\end{frame}	
		
%		\begin{frame} % Chromatin modification
%			
%			\frametitle{Chromatin modifications}
%			\begin{itemize}
%				\item Acetylation, Methylation, Biotinylation
%				\item H3K9me3 and H3K27me3 are markers of silent chromatin
%				\item SETB1 was found to be active after the decline of methylation
%				\item TEs also accumulate when levels of acetylation, biotinylation and ubiquitination are decreased
%			\end{itemize}
%		\end{frame}
		
		\begin{frame} % mRNA decay
			
			\frametitle{mRNA decay}
			\framesubtitle{Targeting the L1 RNA intermediate}
			\begin{columns}
				\begin{column}{0.5\linewidth}
			\begin{itemize}
				\item Targets aberrant transcripts
			%	\item Non-stop decay and no-go decay
				\item Nonsense mediated decay 
				
				
			\end{itemize}
				\end{column}
				\begin{column}{0.5\linewidth}
					\includegraphics[width=\textwidth]{../../Figures/mrna-decay-0.png}
%						\transduration<0-3>{0}
%						\multiinclude[<+->][format=png, graphics={width=\textwidth}]{}
	%				\animatevalue{mrna-decay-}{0}{4}
					%	\animate[loop,controls,width=\textwidth]{12}{mrna-decay-}{0}{4}
				%	\animate
				\end{column}
			\end{columns}
			\begin{itemize}
				\item Alternative splice event $\rightarrow$ Premature Termination Codon $\rightarrow$ Target for NMD
				\item Low coding potential $\rightarrow$ Target for NMD
			\end{itemize}
		\end{frame}	
		
			\section[Alternative Splicing]{Alternative Splicing}
			
		\begin{frame} % A S - Transcription
			\frametitle{mRNA processing}
			\framesubtitle{DNA is transcribed to RNA, which is processed to form mature mRNA}
				
				\begin{figure}
					\centering
					\includegraphics[scale=0.3]{../../Figures/AS-processing.png}
				\end{figure}
			
		\end{frame}	
		
		\begin{frame} % A S - splice forms
			\frametitle{mRNA processing}
			\framesubtitle{Alternative splicing can form multiple splice variants}
			
			\begin{figure}
				\centering
				\includegraphics[scale=0.3]{../../Figures/AS-spliceforms.png}
			\end{figure}
			
		\end{frame}

%		\begin{frame} % Detecting AS
%			\frametitle{Detecting Alternative Splicing}
%			\framesubtitle{Technology for the detection of Alternate splicing has advanced significantly}
%			\begin{itemize}
%				\item dbEST
%				\item Microarray
%				\item RNA-Seq
%				\item \textit{Ab initio} genomic prediction
%			\end{itemize}
%		\end{frame}
		
%		\begin{frame}
			
%			\frametitle{Alternative Splicing in L1s}
				
%				\begin{itemize}
					%\item 
%					We 
%				\end{itemize}
			
			
%		\end{frame}
		
		\section{Project Aims}
		
%		\begin{frame} % Project 
%			\frametitle{Project Motivation}
%			
%			
%			
%			\begin{itemize}
%				\item I will search for AS in L1s
%				\item Why? Because it is an indicator of further processing and regulation
%				\item There is some evidence that L1s are capable of splicing
%			\end{itemize}
%			
%		\end{frame}
		\begin{frame} %Pipeline
			\frametitle{Detecting Alternative Splicing in L1 elements}
			
			\includegraphics[width=0.9\textwidth]{../../Figures/initial-steps.png}
		\end{frame}
		
		\begin{frame} % How a read makes an alignment file
			\frametitle{Detecting Alternative Splicing in L1 elements}
			\framesubtitle{RNA-Seq reads can be aligned to the genome}
			
			\vfill
			\begin{figure}
				\centering
				\includegraphics[width=0.9\linewidth]{../../Figures/RNA-Seq-One-Read.png}
			\end{figure}
			\vfill
			\begin{itemize}
				\item The alignment file will give information about each read
				\item Genome coordinates, read quality
%				\item *We will filter for the L1 coordinates to narrow down reads*
			\end{itemize}
		\end{frame}
			
		\begin{frame} % Many reads making a gene
				\frametitle{Detecting Alternative Splicing in L1 elements}
				\framesubtitle{RNA-Seq reads can be aligned to the genome}
				
				\vfill
				\begin{figure}
					\centering
					\includegraphics[width=0.9\linewidth]{../../Figures/RNA-Seq-complete-read.png}
				\end{figure}
				\vfill
				\begin{itemize}
					\item Reads will overlap, indicating where the L1s are
				\end{itemize}
			\end{frame}
			
		\begin{frame} % Evidence of splicing - Gaps
				\frametitle{Detecting Alternative Splicing in L1 elements}
				\framesubtitle{RNA-Seq reads can be aligned to the genome}
				
				\vfill
				\begin{figure}
					\centering
					\includegraphics[width=0.9\linewidth]{../../Figures/RNA-Seq-AS-evidence.png}
				\end{figure}
				\vfill
				\begin{itemize}
					\item If there is a gap, that will suggest that there has been some splicing, 
				\end{itemize}
			\end{frame}
			
			\begin{frame} % Evidence of splicing - split reads
				\frametitle{Detecting Alternative Splicing in L1 elements}
				\framesubtitle{RNA-Seq reads can be aligned to the genome}
				
				\vfill
				\begin{figure}
					\centering
					\includegraphics[width=0.9\linewidth]{../../Figures/RNA-Seq-AS-split-reads.png}
				\end{figure}
				\vfill
				\begin{itemize}
					\item Reads that are split over two locations on the genome with a gap indicate splicing
				\end{itemize}
			\end{frame}
				
		\begin{frame} %I GV image
			\frametitle{Read visualisation with IGV} 
			\includegraphics[width=1\linewidth]{../../Figures/IGV-1a.png}
		\end{frame}
			
		\begin{frame} % Summary
			\frametitle{Summary}
			
			\begin{itemize}
				\item L1s are the most abundant TEs in the human genome, and we are using them as a candidate for TE regulation
				\item We know they are regulated by a range of mechanisms
				\item To start analysis we are looking for evidence of alternative splicing 
			\end{itemize}
			\includegraphics[width=0.9\textwidth]{../../Figures/initial-steps.png}
		\end{frame}
		
		\begin{frame}
			\frametitle{Further analysis}
			\begin{itemize}
				\item Investigate if the genome itself has alternatively spliced, retrotransposed L1s
				\item Comparative analysis; compare the mouse data with human
			\end{itemize}
		\end{frame}
		
		\begin{frame}
			\frametitle{Further analysis}
			\framesubtitle{If there is no evidence for alternative splicing in the genome}
			\begin{itemize}
				\item Continue with investigation of the genome, 
				\begin{itemize}
					\item Alternative splicing may still be found in the genome, not the transcriptome
				\end{itemize}
				\item Continue with investigation in mice
				\begin{itemize}
					\item Alternative splicing may still occur in L1 transcripts in other organisms
				\end{itemize}	
			\end{itemize}
		\end{frame}		
%			\begin{frame} % 
%				
%				\frametitle{The human genome}
%				\framesubtitle{Repetitive elements are abundant in the human genome}
%				
%				\begin{column}[T]{.7\textwidth}
%					\begin{figure}
%						\includegraphics[scale=0.3]{../../Figures/L1-structure.jpg}
%					\end{figure}
%				\end{column}
%				
%				\begin{column}[T]{.3\textwidth}
%					\begin{itemize}
%						\item 
%					\end{itemize}
%					
%				\end{column}
%			\end{frame}
		
\bibliography{../../literature}
\end{document}